\documentclass{article}

\usepackage{geometry}
\usepackage{amsmath}
\usepackage{amssymb}
\usepackage{amsthm}
\usepackage{mathpartir}
\usepackage{multicol}

\newtheorem*{theorem}{Theorem}
\newtheorem*{prop}{Proposition}
\newtheorem*{definition}{Definition}
\newtheorem*{lemma}{Lemma}
\newtheorem*{example}{Example}
\newtheorem*{remark}{Remark}
\newtheorem*{claim}{Claim}

\renewcommand\qedsymbol{\bf QED}

\geometry{letterpaper, total={7.5in, 10in}, left=.5in, top=.5in}

\setcounter{secnumdepth}{0}

\begin{document}
%%%%%%%%%%%%%%%%%%%%%%%%%%%%%%%%%%%%%%%%%%%%%%%%%%
	\hrule
	\smallskip
	\hrule\hrule\hrule
	\medskip
	\noindent
	{\huge \bf Probabilistic Lambda Calculi}
	\medskip
	\newline
	{\huge \it Research Project}

	\hrule
	\medskip
	\noindent
	{\Large \bf Eric Jackson}
	\medskip
	\hrule\hrule\hrule
	\smallskip
	\hrule
	\medskip
%%%%%%%%%%%%%%%%%%%%%%%%%%%%%%%%%%%%%%%%%%%%%%%%%%
		We define two languages, $\lambda_{\bf IID}$ and $\lambda_{\bf PC}$, for modeling probabilistic behavior. The languages share the same syntax and the same static semantics, but differ in their treatment of the random expressions {\bf coin} and {\bf rand}.
		
	\section{Syntax}
		\begin{multicols}{2}
		\subsection{Expressions}
			\begin{equation*}
				\begin{split}
					e ::=
					&\, x \\
					&| () \\
					&| \lambda x : \tau.\, e \\
					&| \text{\bf let } x = e_1 \text{ \bf in } e_2 \\
					&| e_1\, e_2 \\
					&| r \\
					&| e_1 \oplus e_2 \\
					&| {\bf coin} \\
					&| {\bf rand} \\
					&| (\text{\bf case } e_1 \text{ \bf of } e_2 \,|\, e_3) \\
					&| {\bf inl}_{\tau_1 + \tau_2} e \\
					&| {\bf inr}_{\tau_1 + \tau_2} e \\
					&| \text{\bf \#1 } e \\
					&| \text{\bf \#2 } e \\
					&| (e_1, e_2) \\
				\end{split}
			\end{equation*}
			The languages include a variety of features familiar from standard extensions of the lambda calculus.
			There are two probabilistic constructs: {\bf coin}, which evaluates randomly to $0$ or $1$, and {\bf rand}, which evaluates randomly to a real number in $[0, 1]$.
			Finally, note that $\oplus$ is a placeholder which represents the binary operations: $+$, $-$, $\times$, and $/$.
			
		\subsection{Types}
			\begin{equation*}
				\begin{split}
					\tau ::=
					&\, {\bf unit} \\
					&| \mathbb{R} \\
					&| \tau_1 \to \tau_2 \\
					&| \tau_1 + \tau_2 \\
					&| \tau_1 \times \tau_2 \\
				\end{split}
			\end{equation*}
			The languages includes two base types, {\bf unit} and $\mathbb{R}$---the type of real numbers. 
			Additionally, there are types for functions, sums, and products.
			\end{multicols}
		
	\begin{multicols}{2}
	[
	\section{Static Semantics}
	In addition to sharing the same syntax, the languages we define also share the same static semantics. Their typing rules are listed below.
	]
% Variable
			\begin{equation*}
				\inferrule{
				\inferrule*[right={\sc Var}]
				{\Gamma(x) = \tau}
				{\Gamma \vdash  x : \tau}}
				{}
			\end{equation*}
% Unit
			\begin{equation*}
				\inferrule{
				\inferrule*[right={\sc Unit}]
				{\,}
				{\Gamma \vdash  () : {\bf unit}}}
				{}
			\end{equation*}
% Function
			\begin{equation*}
				\inferrule{
				\inferrule*[right={\sc Fun}]
				{\Gamma \vdash e : \tau'}
				{\Gamma \vdash  (\lambda x : \tau.\, e) : \tau \to \tau'}}
				{}
			\end{equation*}
% Let
			\begin{equation*}
				\inferrule{
				\inferrule*[right={\sc Let}]
				{\Gamma \vdash e_1 : \tau \\ \Gamma_{x : \tau} \vdash e_2 : \tau'}
				{\Gamma \vdash  \text{\bf let } x = e_1 \text{ \bf in } e_2 : \tau'}}
				{}
			\end{equation*}
% Application
			\begin{equation*}
				\inferrule{
				\inferrule*[right={\sc App}]
				{\Gamma \vdash e_1 : \tau \to \tau' \\ \Gamma\vdash e_2 : \tau}
				{\Gamma \vdash e_1\, e_2: \tau'}}
				{}
			\end{equation*}
% Number
			\begin{equation*}
				\inferrule{
				\inferrule*[right={\sc Num}]
				{\,}
				{\Gamma \vdash  r : \mathbb{R}}}
				{}
			\end{equation*}
% Binary Operation
			\begin{equation*}
				\inferrule{
				\inferrule*[right={\sc Bop}]
				{\Gamma \vdash e_1 : \mathbb{R} \\ \Gamma \vdash e_2 : \mathbb{R}}
				{\Gamma \vdash  e_1 \oplus e_2 : \mathbb{R}}}
				{}
			\end{equation*}
% Coin
			\begin{equation*}
				\inferrule{
				\inferrule*[right={\sc Coin}]
				{\,}
				{\Gamma \vdash  {\bf coin} : \mathbb{R}}}
				{}
			\end{equation*}
% Rand
			\begin{equation*}
				\inferrule{
				\inferrule*[right={\sc Rand}]
				{\,}
				{\Gamma \vdash  {\bf rand} : \mathbb{R}}}
				{}
			\end{equation*}
% Case
			\begin{equation*}
				\inferrule{
				\inferrule*[right={\sc Case}]
				{\Gamma \vdash e_1 : \tau_1 + \tau_2 \\ \Gamma \vdash e_2 : \tau_1 \to \tau \\ \Gamma \vdash e_3 : \tau_2 \to \tau }
				{\Gamma \vdash \text{\bf case } e_1 \text{ \bf with } e_2 \,|\, e_3 : \tau}}
				{}
			\end{equation*}
% In-left
			\begin{equation*}
				\inferrule{
				\inferrule*[right={\sc In-Left}]
				{\Gamma \vdash e : \tau_1 }
				{\Gamma \vdash {\bf inl}_{\tau_1 + \tau_2} e : \tau_1 + \tau_2}}
				{}
			\end{equation*}
% In-right
			\begin{equation*}
				\inferrule{
				\inferrule*[right={\sc In-Right}]
				{\Gamma \vdash e : \tau_2 }
				{\Gamma \vdash {\bf inr}_{\tau_1 + \tau_2} e : \tau_1 + \tau_2}}
				{}
			\end{equation*}
% Proj-1
			\begin{equation*}
				\inferrule{
				\inferrule*[right={\sc Proj-1}]
				{\Gamma \vdash e : \tau_1 \times \tau_2 }
				{\Gamma \vdash \text{\bf \# 1 } e : \tau_1}}
				{}
			\end{equation*}
% Proj-2
			\begin{equation*}
				\inferrule{
				\inferrule*[right={\sc Proj-2}]
				{\Gamma \vdash e : \tau_1 \times \tau_2 }
				{\Gamma \vdash \text{\bf \# 2 } e : \tau_2}}
				{}
			\end{equation*}
% Pair
			\begin{equation*}
				\inferrule{
				\inferrule*[right={\sc Pair}]
				{\Gamma \vdash e_1 : \tau_1 \\ \Gamma \vdash e_2 : \tau_2 }
				{\Gamma \vdash (e_1, e_2) : \tau_1 \times \tau_2}}
				{}
			\end{equation*}
	\end{multicols}

	\newpage
	\section{Independently and Identically Distributed ($\lambda_{\bf IID}$)}
		\subsection{Values}
			\begin{equation*}
				\begin{split}
					v ::=
					&\, \lambda x:\tau.\, e \\
					&| () \\
					&| r \\
					&| \text{\bf inl}_{\tau_1 + \tau_2} v \\
					&| \text{\bf inr}_{\tau_1 + \tau_2} v \\
					&| (e, e) \\
				\end{split}
			\end{equation*}
			There are six types of values, which are almost all familiar from traditional lambda calculus extensions.
			Note, however, that in $\lambda_{\bf IID}$ pairs of non-value expressions are values.
			The desirability of this construction can be demonstrated by the expression
			$$\text{\bf let } x = ({\bf coin}, {\bf coin}) \text{ \bf in } (\text {\bf \#1 } x) + (\text{\bf \#1 } x)$$


		\subsection{Semantics}
		The small-step semantics can be modeled as a relation on ${\bf State} \times {\bf State}$ where ${\bf State} \triangleq e \times \{0,1\}^\omega \times [0, 1]^\omega$. This construction allows us to model the behavior of {\bf coin()} and {\bf rand()} in terms of the random sequences in $\{0,1\}^\omega$ and $[0,1]^\omega$, respectively.
		
		We will use a context to facilitate the presentation of the semantics.
		\begin{equation*}
			\begin{split}
				E ::=
				&\, [\cdot] \\
				&| E\, e \\
				&| E \oplus e \\
				&| v \oplus E \\
				&| (\text{\bf case } E \text{ \bf of } e_2 \,|\, e_3) \\
				&| {\bf inl}_{\tau_1 + \tau_2} E \\
				&| {\bf inr}_{\tau_1 + \tau_2} E \\
				&| \text{\bf \# 1 } E \\
				&| \text{\bf \# 2 } E \\
			\end{split}
		\end{equation*}
		
		Finally, the small-step semantics are given by the following inference rules.
		
		
		\begin{multicols}{2}
% Context 
			\begin{equation*}
				\inferrule{
				\inferrule*[right={\sc Context}]
				{\langle e,\, n,\, m \rangle \rightarrow \langle e',\, n',\, m' \rangle }
				{\langle E(e),\, n,\, m\rangle \rightarrow \langle E(e'),\, n',\, m' \rangle}}
				{}
			\end{equation*}
		
% Beta-Reduction
			\begin{equation*}
				\inferrule{
				\inferrule*[right={\sc $\beta$-Reduction}]
				{\,}
				{\langle(\lambda x : \tau.\, e)\, e_2,\, n,\, m\rangle \rightarrow \langle e\{e_2/x\},\, n,\, m \rangle}}
				{}
			\end{equation*}

% Let
			\begin{equation*}
				\inferrule{
				\inferrule*[right={\sc Let}]
				{\,}
				{\langle\text{\bf let } x = e_1 \text{ \bf in } e_2,\, n,\, m\rangle \rightarrow \langle e_2\{e_1/x\},\, n,\, m \rangle}}
				{}
			\end{equation*}

% Binary Operations
			\begin{equation*}
				\inferrule{
				\inferrule*[right={\sc Bop}]
				{r_1 \bar{\oplus} r_2 = r }
				{\langle r_1 \oplus r_2,\, n,\, m\rangle \rightarrow \langle r,\, n,\, m \rangle}}
				{}
			\end{equation*}
		
% Coin
			\begin{equation*}
				\inferrule{
				\inferrule*[right={\sc Coin}]
				{\,}
				{\langle {\bf coin},\, n,\, m\rangle \rightarrow \langle \text{\bf hd } n,\, \text{\bf tl } n,\, m \rangle}}
				{}
			\end{equation*}

% Rand
				\begin{equation*}
				\inferrule{
				\inferrule*[right={\sc Rand}]
				{\,}
				{\langle {\bf rand},\, n,\, m\rangle \rightarrow \langle \text{\bf hd } m,\, n,\, \text{\bf tl } m \rangle}}
				{}
			\end{equation*}

% Case-Left
				\begin{equation*}
				\inferrule{
				\inferrule*[right={\sc Case-Left}]
				{\,}
				{\langle (\text{\bf case } {\bf inl}_{\tau_1 + \tau_2} e \text{ \bf of } e_2 \,|\, e_3) ,\, n,\, m\rangle \rightarrow \langle e_2\, e,\, n,\, m \rangle}}
				{}
			\end{equation*}
% Case-Right
				\begin{equation*}
				\inferrule{
				\inferrule*[right={\sc Case-Right}]
				{\,}
				{\langle (\text{\bf case } {\bf inr}_{\tau_1 + \tau_2} e \text{ \bf of } e_2 \,|\, e_3) ,\, n,\, m\rangle \rightarrow \langle e_3\, e,\, n,\, m \rangle}}
				{}
			\end{equation*}
% Project-1
				\begin{equation*}
				\inferrule{
				\inferrule*[right={\sc Proj-1}]
				{\,}
				{\langle \text{\bf \#1 } (e_1, e_2) ,\, n,\, m\rangle \rightarrow \langle e_1,\, n,\, m \rangle}}
				{}
			\end{equation*}
% Project-2
				\begin{equation*}
				\inferrule{
				\inferrule*[right={\sc Proj-2}]
				{\,}
				{\langle \text{\bf \#2 } (e_1, e_2) ,\, n,\, m\rangle \rightarrow \langle e_2,\, n,\, m \rangle}}
				{}
			\end{equation*}
		\end{multicols}


	\subsection{Example}
			\begin{equation*}
				\begin{split}
					\langle (\lambda x.\, x + x )\, {\bf coin},\, 1::0::n,\, m\rangle
					&\rightarrow \langle {\bf coin} + {\bf coin},\, 1::0::n,\, m\rangle \\
					&\rightarrow \langle 1 + {\bf coin},\, 0::n,\, m\rangle \\
					&\rightarrow \langle 1 + 0,\, n,\, m\rangle \\
					&\rightarrow \langle 1,\, n,\, m\rangle \\
				\end{split}
			\end{equation*}
			
	\newpage
	\section{Perfectly Correlated ($\lambda_{\bf PC}$)}
		\subsection{Values}
			\begin{equation*}
				\begin{split}
					v ::=
					&\, \lambda x:\tau.\, e \\
					&| () \\
					&| r \\
					&| \text{\bf inl}_{\tau_1 + \tau_2} v \\
					&| \text{\bf inr}_{\tau_1 + \tau_2} v \\
					&| (v, v) \\
				\end{split}
			\end{equation*}
			Again, there are six types of values. 
			Note, however, that unlike $\lambda_{\bf IID}$ a pair is only a value if it is a pair of values.
		
		\subsection{Semantics}

		As with $\lambda_{\bf IID}$, the small-step semantics of $\lambda_{\bf PC}$ can be modeled as a relation on $\bf State \times State$.
		
		We will again use a context to facilitate presentation, but one should notice several differences between the contexts of the two languages. 
		For example, the $\lambda_{\bf PC}$ context includes pairs.
		
		\begin{equation*}
			\begin{split}
				E ::=
				&\, [\cdot] \\
				&| E\, e \\
				&| v\, E \\
				&| \text{\bf let } x = E \text{ \bf in } e_2 \\
				&| E \oplus e \\
				&| v \oplus E \\
				&| (\text{\bf case } E \text{ \bf of } e_2 \,|\, e_3) \\
				&| {\bf inl}_{\tau_1 + \tau_2} E \\
				&| {\bf inr}_{\tau_1 + \tau_2} E \\
				&| \text{\bf \# 1 } E \\
				&| \text{\bf \# 2 } E \\
				&| (E, e) \\
				&| (v, E) \\
			\end{split}
		\end{equation*}
		
		The small-step semantics are given by the following inference rules.
		
		\begin{multicols}{2}
% Context 
			\begin{equation*}
				\inferrule{
				\inferrule*[right={\sc Context}]
				{\langle e,\, n,\, m \rangle \rightarrow \langle e',\, n',\, m' \rangle }
				{\langle E(e),\, n,\, m\rangle \rightarrow \langle E(e'),\, n',\, m' \rangle}}
				{}
			\end{equation*}
		
% Beta-Reduction
			\begin{equation*}
				\inferrule{
				\inferrule*[right={\sc $\beta$-Reduction}]
				{\,}
				{\langle(\lambda x : \tau.\, e)\, v,\, n,\, m\rangle \rightarrow \langle e\{v/x\},\, n,\, m \rangle}}
				{}
			\end{equation*}

% Let
			\begin{equation*}
				\inferrule{
				\inferrule*[right={\sc Let}]
				{\,}
				{\langle\text{\bf let } x = v \text{ \bf in } e_2,\, n,\, m\rangle \rightarrow \langle e_2\{v/x\},\, n,\, m \rangle}}
				{}
			\end{equation*}

% Binary Operations
			\begin{equation*}
				\inferrule{
				\inferrule*[right={\sc Bop}]
				{r_1 \bar{\oplus} r_2 = r }
				{\langle r_1 \oplus r_2,\, n,\, m\rangle \rightarrow \langle r,\, n,\, m \rangle}}
				{}
			\end{equation*}
		
% Coin
			\begin{equation*}
				\inferrule{
				\inferrule*[right={\sc Coin}]
				{\,}
				{\langle {\bf coin},\, n,\, m\rangle \rightarrow \langle \text{\bf hd } n,\, \text{\bf tl } n,\, m \rangle}}
				{}
			\end{equation*}

% Rand
				\begin{equation*}
				\inferrule{
				\inferrule*[right={\sc Rand}]
				{\,}
				{\langle {\bf rand},\, n,\, m\rangle \rightarrow \langle \text{\bf hd } m,\, n,\, \text{\bf tl } m \rangle}}
				{}
			\end{equation*}

% Case-Left
				\begin{equation*}
				\inferrule{
				\inferrule*[right={\sc Case-Left}]
				{\,}
				{\langle (\text{\bf case } {\bf inl}_{\tau_1 + \tau_2} v \text{ \bf of } e_2 \,|\, e_3) ,\, n,\, m\rangle \rightarrow \langle e_2\, v,\, n,\, m \rangle}}
				{}
			\end{equation*}
% Case-Right
				\begin{equation*}
				\inferrule{
				\inferrule*[right={\sc Case-Right}]
				{\,}
				{\langle (\text{\bf case } {\bf inr}_{\tau_1 + \tau_2} v \text{ \bf of } e_2 \,|\, e_3) ,\, n,\, m\rangle \rightarrow \langle e_3\, v,\, n,\, m \rangle}}
				{}
			\end{equation*}
% Project-1
				\begin{equation*}
				\inferrule{
				\inferrule*[right={\sc Proj-1}]
				{\,}
				{\langle \text{\bf \#1 } (v_1, v_2) ,\, n,\, m\rangle \rightarrow \langle v_1,\, n,\, m \rangle}}
				{}
			\end{equation*}
% Project-2
				\begin{equation*}
				\inferrule{
				\inferrule*[right={\sc Proj-2}]
				{\,}
				{\langle \text{\bf \#2 } (v_1, v_2) ,\, n,\, m\rangle \rightarrow \langle v_2,\, n,\, m \rangle}}
				{}
			\end{equation*}
		\end{multicols}

	\subsection{Example}
			\begin{equation*}
				\begin{split}
					\langle \text{\bf let } x = {\bf coin} \text{ \bf in } x + x,\, 1::0::n,\, m\rangle
					&\rightarrow \langle \text{\bf let } x = 1 \text{ \bf in } x + x,\, 0::n,\, m\rangle \\
					&\rightarrow \langle 1 + 1,\, 0::n,\, m\rangle \\
					&\rightarrow \langle 2,\, 0::n,\, m\rangle \\
				\end{split}
			\end{equation*}

\end{document}