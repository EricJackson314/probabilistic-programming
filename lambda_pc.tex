\documentclass{article}

\usepackage{geometry}
\usepackage{amsmath}
\usepackage{amssymb}
\usepackage{amsthm}
\usepackage{mathpartir}
\usepackage{multicol}

\newtheorem*{theorem}{Theorem}
\newtheorem*{prop}{Proposition}
\newtheorem*{definition}{Definition}
\newtheorem*{lemma}{Lemma}
\newtheorem*{example}{Example}
\newtheorem*{remark}{Remark}
\newtheorem*{claim}{Claim}

\renewcommand\qedsymbol{\bf QED}

\geometry{letterpaper, total={7.5in, 10in}, left=.5in, top=.5in}

\setcounter{secnumdepth}{0}

\begin{document}
%%%%%%%%%%%%%%%%%%%%%%%%%%%%%%%%%%%%%%%%%%%%%%%%%%
	\hrule
	\smallskip
	\hrule\hrule\hrule
	\medskip
	\noindent
	{\huge \bf Perfectly Correlated}
	\medskip
	\newline
	{\huge \it $\lambda_{PC}$}
	\smallskip
	\hrule
	\medskip
	\noindent
	{\Large \bf Eric Jackson}
	\medskip
	\hrule\hrule\hrule
	\smallskip
	\hrule
%%%%%%%%%%%%%%%%%%%%%%%%%%%%%%%%%%%%%%%%%%%%%%%%%%

% Syntax %%%%%%%%%%%%%%%%%%%%%%%%%%%%%%%%%%%%%%%%%
\section{Syntax}

TODO

% Static Semantics %%%%%%%%%%%%%%%%%%%%%%%%%%%%%%%
\section{Static Semantics}

TODO

% Dynamic Semantics %%%%%%%%%%%%%%%%%%%%%%%%%%%%%%
\section{Dynamic Semantic}

% Small Step %%%%%%%%%%%%%%%%%%%%%%%%%%%%%%%%%%%%%
	\subsection{Small-Step Semantics}
	\begin{multicols}{2}
% Context 
			\begin{equation*}
				\inferrule{
				\inferrule*[right={\sc Context}]
				{\langle e,\, n,\, m \rangle \rightarrow \langle e',\, n',\, m' \rangle }
				{\langle E(e),\, n,\, m\rangle \rightarrow \langle E(e'),\, n',\, m' \rangle}}
				{}
			\end{equation*}
		
% Beta-Reduction
			\begin{equation*}
				\inferrule{
				\inferrule*[right={\sc $\beta$-Reduction}]
				{\,}
				{\langle(\lambda x : \tau.\, e)\, v,\, n,\, m\rangle \rightarrow \langle e\{v/x\},\, n,\, m \rangle}}
				{}
			\end{equation*}

% Let
			\begin{equation*}
				\inferrule{
				\inferrule*[right={\sc Let}]
				{\,}
				{\langle\text{\bf let } x = v \text{ \bf in } e_2,\, n,\, m\rangle \rightarrow \langle e_2\{v/x\},\, n,\, m \rangle}}
				{}
			\end{equation*}

% Binary Operations
			\begin{equation*}
				\inferrule{
				\inferrule*[right={\sc Bop}]
				{r_1 \bar{\oplus} r_2 = r }
				{\langle r_1 \oplus r_2,\, n,\, m\rangle \rightarrow \langle r,\, n,\, m \rangle}}
				{}
			\end{equation*}
		
% Coin
			\begin{equation*}
				\inferrule{
				\inferrule*[right={\sc Coin}]
				{\,}
				{\langle {\bf coin()},\, n,\, m\rangle \rightarrow \langle \text{\bf hd } n,\, \text{\bf tl } n,\, m \rangle}}
				{}
			\end{equation*}

% Rand
				\begin{equation*}
				\inferrule{
				\inferrule*[right={\sc Rand}]
				{\,}
				{\langle {\bf rand()},\, n,\, m\rangle \rightarrow \langle \text{\bf hd } m,\, n,\, \text{\bf tl } m \rangle}}
				{}
			\end{equation*}

% Case-Left
				\begin{equation*}
				\inferrule{
				\inferrule*[right={\sc Case-Left}]
				{\,}
				{\langle (\text{\bf case } {\bf inl}_{\tau_1 + \tau_2} v \text{ \bf of } e_2 \,|\, e_3) ,\, n,\, m\rangle \rightarrow \langle e_2\, v,\, n,\, m \rangle}}
				{}
			\end{equation*}
% Case-Right
				\begin{equation*}
				\inferrule{
				\inferrule*[right={\sc Case-Right}]
				{\,}
				{\langle (\text{\bf case } {\bf inr}_{\tau_1 + \tau_2} v \text{ \bf of } e_2 \,|\, e_3) ,\, n,\, m\rangle \rightarrow \langle e_3\, v,\, n,\, m \rangle}}
				{}
			\end{equation*}
% Project-1
				\begin{equation*}
				\inferrule{
				\inferrule*[right={\sc Proj-1}]
				{\,}
				{\langle \text{\bf \#1 } (v_1, v_2) ,\, n,\, m\rangle \rightarrow \langle v_1,\, n,\, m \rangle}}
				{}
			\end{equation*}
% Project-2
				\begin{equation*}
				\inferrule{
				\inferrule*[right={\sc Proj-2}]
				{\,}
				{\langle \text{\bf \#2 } (v_1, v_2) ,\, n,\, m\rangle \rightarrow \langle v_2,\, n,\, m \rangle}}
				{}
			\end{equation*}
% To
			\begin{equation*}
				\inferrule{
				\inferrule*[right={\sc To-iid}]
				{\,}
				{\langle e_1 \text{ \bf to } x \text{ \bf in } e_2, n, m \rangle \to 
				\langle e_2 \{ (\lambda y.\, e_1) \, () / x \}, n, m \rangle}}
				{}
			\end{equation*}
		\end{multicols}
			
% Big Step %%%%%%%%%%%%%%%%%%%%%%%%%%%%%%%%%%%%%%%
	\subsection{Big-Step Semantics}
% Beta-Reduction
			\begin{equation*}
				\inferrule{
				\inferrule*[right={\sc $\beta$-Reduction}]
				{\langle e_1, n, m \rangle \Downarrow \langle \lambda x.\, e_1', n', m'\rangle \\
				\langle e_2, n', m' \rangle \Downarrow \langle v, n'', m''\rangle \\
				\langle e_1'\{v/x\}, n'',m''\rangle \Downarrow \langle v', n''', m''' \rangle}
				{\langle e_1\, e_2,\, n,\, m\rangle \Downarrow \langle v',\, n''',\, m''' \rangle}}
				{}
			\end{equation*}

% Let
			\begin{equation*}
				\inferrule{
				\inferrule*[right={\sc Let}]
				{\langle e_1, n, m\rangle \Downarrow \langle v, n', m'\rangle \\
				\langle e_2\{v/x\}, n', m' \rangle \Downarrow \langle v', n'', m''\rangle }
				{\langle\text{\bf let } x = e_1 \text{ \bf in } e_2,\, n,\, m\rangle \Downarrow \langle v',\, n'',\, m'' \rangle}}
				{}
			\end{equation*}

% Binary Operations
			\begin{equation*}
				\inferrule{
				\inferrule*[right={\sc Bop}]
				{\langle e_1, n, m \rangle \Downarrow \langle r_1, n', m' \rangle \\
				\langle e_2, n', m' \rangle \Downarrow \langle r_2, n'', m'' \rangle \\
				 r_1 \bar{\oplus} r_2 = r }
				{\langle e_1 \oplus e_2,\, n,\, m\rangle \Downarrow \langle r,\, n'',\, m'' \rangle}}
				{}
			\end{equation*}
% Coin
			\begin{equation*}
				\inferrule{
				\inferrule*[right={\sc Coin}]
				{\,}
				{\langle {\bf coin},\, n,\, m\rangle \Downarrow \langle \text{\bf hd } n,\, \text{\bf tl } n,\, m \rangle}}
				{}
			\end{equation*}

% Rand
				\begin{equation*}
				\inferrule{
				\inferrule*[right={\sc Rand}]
				{\,}
				{\langle {\bf rand},\, n,\, m\rangle \Downarrow \langle \text{\bf hd } m,\, n,\, \text{\bf tl } m \rangle}}
				{}
			\end{equation*}
% Case-Left
				\begin{equation*}
				\inferrule{
				\inferrule*[right={\sc Case-Left}]
				{\langle e_2\, e_1, n, m \rangle \Downarrow \langle v, n', m' \rangle}
				{\langle (\text{\bf case } {\bf inl}_{\tau_1 + \tau_2} e_1 \text{ \bf of } e_2 \,|\, e_3) ,\, n,\, m\rangle \Downarrow \langle v,\, n',\, m' \rangle}}
				{}
			\end{equation*}

% Case-Right
				\begin{equation*}
				\inferrule{
				\inferrule*[right={\sc Case-Left}]
				{\langle e_3\, e_1, n, m \rangle \Downarrow \langle v, n', m' \rangle}
				{\langle (\text{\bf case } {\bf inr}_{\tau_1 + \tau_2} e_1 \text{ \bf of } e_2 \,|\, e_3) ,\, n,\, m\rangle \Downarrow \langle v,\, n',\, m' \rangle}}
				{}
			\end{equation*}

% Project-1
				\begin{equation*}
				\inferrule{
				\inferrule*[right={\sc Proj-1}]
				{\langle e_1, n, m \rangle \Downarrow \langle v, n', m' \rangle}
				{\langle \text{\bf \#1 } (e_1, e_2) ,\, n,\, m\rangle \Downarrow \langle v, n', m' \rangle}}
				{}
			\end{equation*}
			
% Project-2
				\begin{equation*}
				\inferrule{
				\inferrule*[right={\sc Proj-2}]
				{\langle e_2, n, m \rangle \Downarrow \langle v, n', m' \rangle}
				{\langle \text{\bf \#2 } (e_1, e_2) ,\, n,\, m\rangle \Downarrow \langle v, n', m' \rangle}}
				{}
			\end{equation*}
			
% To
			\begin{equation*}
				\inferrule{
				\inferrule*[right={\sc To-iid}]
				{\langle e_2\{ (\lambda y. e_1) ()/ x\}, n, m \rangle \Downarrow \langle v, n' m' \rangle}
				{\langle e_1 \text{ \bf to } x \text{ \bf in } e_2, n, m \rangle  
				\Downarrow \langle v, n', m' \rangle}}
				{}
			\end{equation*}


% Denotational %%%%%%%%%%%%%%%%%%%%%%%%%%%%%%%%%%%
	\subsection{Denotational Semantics}
		\begin{equation*}
		\begin{split}
			[\![r]\!] &\triangleq \delta_r \\
			[\![\lambda x.\, e]\!] &\triangleq  \\
			[\![{\bf coin}]\!] &\triangleq \\
			[\![e_1 \text{ } e_2]\!] &\triangleq \\
		\end{split}
	\end{equation*}

\end{document}