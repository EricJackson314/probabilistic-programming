\documentclass[12pt,a4paper]{article}
\usepackage{amsfonts}
\usepackage{amsthm}
\usepackage{amsmath}
\usepackage{amscd}
\usepackage{bbold}
\usepackage{ amssymb }
\usepackage{mathtools}
\usepackage{mathpartir}
\usepackage[latin2]{inputenc}
\usepackage{t1enc}
\usepackage[mathscr]{eucal}
\usepackage{indentfirst}
\usepackage{graphicx}
\usepackage{graphics}
\usepackage{pict2e}
\usepackage{epic}
\usepackage{syntax}
\usepackage{enumitem}
\usepackage[colorlinks]{hyperref}
\numberwithin{equation}{section}
\usepackage[margin=2.9cm]{geometry}
\usepackage{epstopdf}

\usepackage{tikz-cd}

\usepackage{ stmaryrd }

\newcommand{\cat}{%
\mathbf%
}

\newcommand{\sem}[1]{
  \llbracket #1 \rrbracket
}


\def\numset#1{{\\mathbb #1}}


   
 

\theoremstyle{plain}
\newtheorem{Th}{Theorem}[section]
\newtheorem{Lemma}[Th]{Lemma}
\newtheorem{Cor}[Th]{Corollary}
\newtheorem{Prop}[Th]{Proposition}
\newtheorem{Q}[Th]{Question}

\theoremstyle{definition}
\newtheorem{Def}[Th]{Definition}
\newtheorem{Conj}[Th]{Conjecture}
\newtheorem{Rem}[Th]{Remark}
\newtheorem{?}[Th]{Problem}
\newtheorem{Ex}[Th]{Example}

\newcommand{\im}{\operatorname{im}}
\newcommand{\Hom}{{\rm{Hom}}}
\newcommand{\diam}{{\rm{diam}}}
\newcommand{\ovl}{\overline}
\newcommand{\norm}[1]{||#1||}
\newcommand{\test}[1]{\mathcal D(#1)}
\newcommand{\dist}[1]{\mathcal D'(#1)}
\newcommand{\diff}[1]{(#1, \mathcal P^U_{#1})}



\newcommand{\set}[2]{\{#1 \, | \, #2 \}}

\newcommand{\lrang}[1]{ \langle #1 \rangle }


\newcommand{\M}{\mathcal{M}}


%Grammar
\newcommand{\smid}{\ \mid \ }
\newcommand{\defined}{\coloneqq}
\newcommand{\subst}[3]{#1\{#2 / #3\}}

%% types
\newcommand{\nat}{\mathbb{N}}
\newcommand{\R}{\mathbb{R}}

%% stlc
\newcommand{\lamb}[2]{\lambda #1. \ #2}
\newcommand{\app}[2]{#1 \ #2}

%% cbpv
\newcommand{\produce}{\mathsf{produce}\ }
\newcommand{\thunk}{\mathsf{thunk}\ }
\newcommand{\force}{\mathsf{force}\ }
\newcommand{\letbe}[3]{\mathsf{let}\ #1 \ \mathsf{be}\ #2 \ \mathsf{in}\ #3}
\newcommand{\mtox}[3]{#1 \ \mathsf{to} \ #2 . \ #3}
\newcommand{\consume}{\mathsf{consume}\ }
\newcommand{\supply}{\mathsf{supply}\ }

%% probabilistic programming
\newcommand{\rand}{\mathsf{rand}}


% Typing rule environment

\newenvironment{typing}
{\begin{center}
    \begin{tabular}{c c}
      \begin{mathpar}
}
{
   \end{mathpar}
  \end{tabular}
\end{center}
}




\begin{document}

\title{A probabilistic CPBV}

\author{Pedro Amorim}

\maketitle

\section{Syntax}

The type grammar is the same as the non-probabilistic variant of CBPV:

\begin{align*}
  & \overline{\tau} \defined F \tau \smid \tau \to \overline \tau \\
  & \tau \defined U \overline{\tau} \smid 1 \smid \tau \times \tau \smid \R\\
\end{align*}


The syntax and typing rules are also very similar to the regular CBPV expect for:

\begin{typing}
  \inferrule{ }{\Gamma \vdash^c \rand : F \R}
\end{typing}

The program $\rand$ represents the uniform measure over the interval $[ 0, 1]$ and being an effectful operation, requires the type constructor $F$.

\section{Operational semantics}

To capture the fact that programs denote measures, our operational semantics will also have an stochastic flavor: the small-step operational semantics is a markov kernel (generalized transition matrix).

We say that $t \rightarrow \mu$ if the program $t$, after one step of computation, evaluates to the measure over terms $\mu$. Note that since we have a type for real numbers, we require a non-trivial $\sigma$-algebra over the set of terms.

The two interesting reduction rules are the ones for $\rand$ and the context rules:

\begin{typing}

  \inferrule{ }{\rand \rightarrow \mu}

  \and

  \inferrule{t \rightarrow \mu_t}{E[t] \rightarrow \mu_t^*(t \mapsto E[t]) }
  
\end{typing}

Where $\mu$ is the lebesgue uniform measure and $\mu^*(f)$ is the pushforward measure of $\mu$ along the measurable function $f$.

To define multiple reductions of a term we simply make use of Kernel composition. We need to prove that everything is measurable, but that's not terribly difficult to do and has been done before.



\end{document}