\documentclass{article}

\usepackage{geometry}
\usepackage{amsmath}
\usepackage{amssymb}
\usepackage{amsthm}
\usepackage{mathpartir}
\usepackage{multicol}

\newtheorem*{theorem}{Theorem}
\newtheorem*{prop}{Proposition}
\newtheorem*{definition}{Definition}
\newtheorem*{lemma}{Lemma}
\newtheorem*{example}{Example}
\newtheorem*{remark}{Remark}
\newtheorem*{claim}{Claim}

% math
\renewcommand\qedsymbol{\bf QED}
\DeclareMathOperator*\R{\mathbb{R}}
\DeclareMathOperator*\Q{\mathbb{Q}}
\DeclareMathOperator*\Z{\mathbb{Z}}
\DeclareMathOperator*\N{\mathbb{N}}
\DeclareMathOperator*\llb{[\![}
\DeclareMathOperator*\rrb{]\!]}

% syntax
\newcommand{\rand}{\mathsf{rand}}

\geometry{letterpaper, total={7.5in, 10in}, left=.5in, top=.5in}

\begin{document}
%%%%%%%%%%%%%%%%%%%%%%%%%%%%%%%%%%%%%%%%%%%%%%%%%%
	\hrule
	\smallskip
	\hrule\hrule\hrule
	\medskip
	\noindent
	{\huge \bf Probabilistic Lambda Calculi}
	\medskip
	\newline
	{\huge \it Research Project}

	\hrule
	\medskip
	\noindent
	{\Large \bf Pedro Amorim and Eric Jackson}
	\medskip
	\hrule\hrule\hrule
	\smallskip
	\hrule
%%%%%%%%%%%%%%%%%%%%%%%%%%%%%%%%%%%%%%%%%%%%%%%%%%

\section{Abstract}

In section 2, we introduce a probabilistic version of Call-By-Push Value.
In section 3, we introduce two probabilistic lambda-calculi,
$\lambda_{IID}$ and $\lambda_{PC}$, which correspond to 
independent and identically distributed sampling 
and perfectly correlated sampling, respectively.

\section{Probabilistic Call-By-Push-Value}
\subsection{Syntax}
\subsubsection*{Types}
	CBPV types are given by $\tau = A \,|\, B$
	where $A$ and $B$ are defined inductively according to the following rules:
	\begin{equation*}
		\begin{split}
			A &::= 
			U \underline{B} 
			\,|\, A + A
			\,|\, A \times A
			\,|\, 1			
			\,|\, \R \\
			\underline{B} &::=
			F A
			\,|\, \underline{B} \times \underline{B}
			\,|\, A \to \underline{B} \\
		\end{split}
	\end{equation*}
	
\subsubsection*{Expressions}
	\begin{equation*}
	\end{equation*}

\subsection{Static Semantics}
	\inferrule{ }{\Gamma \vdash^c \rand : F \R}

\subsection{Denotational Semantics}

\section{$\lambda_{IID}$ and $\lambda_{PC}$}

\subsection{Syntax}

\section{Translating $\lambda_{IID}$ and $\lambda_{PC}$ to CBPV}
Rather than defining denotational semantics for $\lambda_{IID}$ and $\lambda_{PC}$,
we will define a translation from each of these languages into CBPV.
Then, we can use the CBPV semantics to generate denotational semantics.

\subsection{Types}
We begin with defining type translations from $\lambda_{IID}$ and $\lambda_{PC}$ to CBVP.
These translations are essentially identical to the translations of CBN and CBV to CBPV presented in Levy's thesis.

\begin{multicols}{2}
\subsubsection*{$\lambda_{IID}$}
\hrule\hrule
\medskip
	\begin{equation*}
		\begin{split}
			\llb {\bf unit} {\rrb}_{IID} &\triangleq F 1 \\
			\llb \R {\rrb}_{IID} &\triangleq F \R \\
			\llb \tau_1 \to \tau_2 {\rrb}_{IID} &\triangleq (U \llb \tau_1 {\rrb}_{IID}) \to \llb \tau_2 {\rrb}_{IID}\\
			\llb \tau_1 + \tau_2 {\rrb}_{IID} &\triangleq F(U\llb \tau_1 {\rrb}_{IID} + U\llb \tau_2 {\rrb}_{IID})\\
			\llb \tau_1 \times \tau_2 {\rrb}_{IID} &\triangleq \llb \tau_1 {\rrb}_{IID} \times \llb \tau_2 {\rrb}_{IID}\\
		\end{split}
	\end{equation*}
		
\subsubsection*{$\lambda_{PC}$}
\hrule\hrule
\medskip
	\begin{equation*}
		\begin{split}
			\llb {\bf unit} {\rrb}_{PC} &\triangleq 1 \\
			\llb \R {\rrb}_{PC} &\triangleq \R \\
			\llb \tau_1 \to \tau_2 {\rrb}_{PC} &\triangleq U (\llb \tau_1 {\rrb}_{PC} \to \llb F\tau_2 {\rrb}_{PC})\\
			\llb \tau_1 + \tau_2 {\rrb}_{PC} &\triangleq U\llb \tau_1 {\rrb}_{PC} + \llb \tau_2 {\rrb}_{PC}\\
			\llb \tau_1 \times \tau_2 {\rrb}_{PC} &\triangleq U(F\llb\tau_1{\rrb}_{PC} \times F\llb\tau_2{\rrb}_{PC})\\
		\end{split}
	\end{equation*}
\end{multicols}

\subsection{Expressions}
Now, we define translations from expressions in $\lambda_{IID}$ and $\lambda_{PC}$
to CBPV.
\subsubsection*{$\lambda_{IID}$}
\hrule\hrule
\medskip
	\begin{equation*}
		\begin{split}
			\mathcal{T} [\![x{\rrb}_{IID} &\triangleq 
			\text{\bf force } x \\
			\mathcal{T} [\![\lambda x.\, e{\rrb}_{IID} &\triangleq 
			\lambda x.\, \mathcal[\![ e {\rrb}_{IID} \\
			\mathcal{T} [\![\text{\bf let} x = e_1 \text{ \bf in } e_2{\rrb}_{IID} &\triangleq 
			\text{\bf let } x \text{ \bf be } \mathcal{T}[\![e_1{\rrb}_{IID}.\, \mathcal{T}[\![e_2{\rrb}_{IID}\\
			\mathcal{T} [\![e_1\, e_2 {\rrb}_{IID} &\triangleq 
			(\text{\bf thunk } \mathcal{T}[\![e_2{\rrb}_{IID})`\, \mathcal{T}[\![e_1{\rrb}_{IID}\\
			\mathcal{T} [\![\text{\bf coin}{\rrb}_{IID} &\triangleq
			\text{\bf coin} \\
			\mathcal{T} [\![\text{\bf rand}{\rrb}_{IID} &\triangleq
			\text{\bf rand} \\
			\mathcal{T} [\![\text{\bf inl}_{\tau_1 + \tau_2} e{\rrb}_{IID} &\triangleq
			\text{\bf produce inl thunk } \mathcal{T}[\![e{\rrb}_{IID} \\
			\mathcal{T} [\![\text{\bf inr}_{\tau_1 + \tau_2} e{\rrb}_{IID} &\triangleq
			\text{\bf produce inr thunk } \mathcal{T}[\![e{\rrb}_{IID} \\
			\mathcal{T} [\![\text{\bf case } e_1 \text{\bf of } e_2 | e_3{\rrb}_{IID} &\triangleq	
			\mathcal{T}[\![ e_1{\rrb}_{IID} \text{ \bf to } z. \text{ \bf pm } z \text{ \bf as } 
			\{\text{\bf inl } x. \mathcal{T}[\![ e_2{\rrb}_{IID}, 
			\text{ \bf inr } x. \mathcal{T}[\![ e_3{\rrb}_{IID}\} \\
			\mathcal{T} [\![(e_1, e_2){\rrb}_{IID} &\triangleq \\
			\mathcal{T} [\![\text{\bf \#1 } e{\rrb}_{IID} &\triangleq \\
			\mathcal{T} [\![\text{\bf \#2 } e{\rrb}_{IID} &\triangleq \\
			\mathcal{T} [\![e_1 \text{ \bf to } x \text{ \bf in } e2{\rrb}_{IID} &\triangleq \\
		\end{split}
	\end{equation*}
	
		
\subsubsection*{$\lambda_{PC}$}	
\hrule\hrule
\medskip
	\begin{equation*}
		\begin{split}
			\mathcal{T} [\![x{\rrb}_{PC} &\triangleq 
			\text{\bf produce } x \\
			\mathcal{T} [\![\lambda x.\, e{\rrb}_{PC} &\triangleq 
			\text{\bf produce thunk } \lambda x.\, \mathcal[\![ e {\rrb}_{PC} \\
			\mathcal{T} [\![\text{\bf let} x = e_1 \text{ \bf in } e_2{\rrb}_{PC} &\triangleq 
			\mathcal{T}[\![e_1{\rrb}_{PC} \text{ \bf to } x.\, \mathcal{T}[\![e_2{\rrb}_{PC}\\
			\mathcal{T} [\![e_1\, e_2{\rrb}_{PC} &\triangleq 
			\mathcal{T}[\![e_2{\rrb}_{PC} \text{ \bf to } x.\, \mathcal{T}[\![e_1{\rrb}_{PC} \text{ \bf to } f.\, x` (\text{\bf force } f)\\
			\mathcal{T} [\![\text{\bf coin}{\rrb}_{PC} &\triangleq 
			\text{\bf produce coin} \\
			\mathcal{T} [\![\text{\bf rand}{\rrb}_{PC} &\triangleq
			\text{\bf produce rand} \\
			\mathcal{T} [\![\text{\bf inl}_{\tau_1 + \tau_2} e{\rrb}_{PC} &\triangleq
			\mathcal{T}[\![e{\rrb}_{PC} \text{ \bf to } z.\, \text{\bf produce inl } z \\
			\mathcal{T} [\![\text{\bf inr}_{\tau_1 + \tau_2} e{\rrb}_{PC} &\triangleq
			\mathcal{T}[\![e{\rrb}_{PC} \text{ \bf to } z.\, \text{\bf produce inr } z \\
			\mathcal{T} [\![\text{\bf case } e_1 \text{\bf of } e_2 | e_3{\rrb}_{PC} &\triangleq
			\mathcal{T}[\![ e_1{\rrb}_{PC} \text{ \bf to } z. \text{ \bf pm } z \text{ \bf as } 
			\{ \text{\bf inl } x. \mathcal{T}[\![ e_2{\rrb}_{PC}, \text{ \bf inr } x. \mathcal{T}[\![ e_3{\rrb}_{PC}\} \\
			\mathcal{T} [\![(e_1, e_2){\rrb}_{PC} &\triangleq \\
			\mathcal{T} [\![\text{\bf \#1 } e{\rrb}_{PC} &\triangleq \\
			\mathcal{T} [\![\text{\bf \#2 } e{\rrb}_{PC} &\triangleq \\
			\mathcal{T} [\![e_1 \text{ \bf to } x \text{ \bf in } e2{\rrb}_{PC} &\triangleq \\
		\end{split}
	\end{equation*}

\begin{theorem}
	If $\Gamma \vdash_{IID} e : \tau$ 
	then $$\Gamma \vdash \llb e {\rrb}_{IID} : {\llb \tau\rrb}_{IID}$$
	and if $\Gamma \vdash_{PC} e : \tau$ 
	then $$\Gamma \vdash \llb e {\rrb}_{PC} : {\llb \tau\rrb}_{PC}$$
\end{theorem}

\begin{proof}
First, assume $\Gamma \vdash_{IID} e : \tau$.
We proceed by induction on $e$.
\begin{itemize}
	\item {\bf Case:} $e = x$ ... TODO!
	
	\item {\bf Case:} $e = \lambda x : \tau .\, e'$.
	By inspection, $\Gamma \vdash_{IID} e : \tau \to \tau'$ for some $\tau'$ such that
	$\Gamma_{x : \tau} \vdash_{IID}  e' : \tau'$.
	Then $\llb e {\rrb}_{IID} = \lambda x. \llb e' \rrb_{IID}$.
	
	\item TODO!
\end{itemize}

Now, assume $\Gamma \vdash_{PC} e : \tau$. TODO!
\end{proof}

\begin{theorem}
	If $\Gamma \vdash_{IID} e : \tau_1 \times \tau_2$
	then $\llb e \rrb = \mu_1 \times \mu_2$.
\end{theorem}

\begin{proof}
TODO!
\end{proof}

\end{document}