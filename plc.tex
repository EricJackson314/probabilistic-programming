\documentclass{article}

\usepackage{geometry}
\usepackage{amsmath}
\usepackage{amssymb}
\usepackage{amsthm}
\usepackage{mathpartir}
\usepackage{multicol}

\newtheorem*{theorem}{Theorem}
\newtheorem*{prop}{Proposition}
\newtheorem*{definition}{Definition}
\newtheorem*{lemma}{Lemma}
\newtheorem*{example}{Example}
\newtheorem*{remark}{Remark}
\newtheorem*{claim}{Claim}

% math
\renewcommand\qedsymbol{\bf QED}
\DeclareMathOperator*\R{\mathbb{R}}
\DeclareMathOperator*\Q{\mathbb{Q}}
\DeclareMathOperator*\Z{\mathbb{Z}}
\DeclareMathOperator*\N{\mathbb{N}}
\DeclareMathOperator*\llb{[\![}
\DeclareMathOperator*\rrb{]\!]}

% languages
\newcommand{\iid}{\mathsf{IID}}
\newcommand{\pc}{\mathsf{PC}}

% syntax
\newcommand{\x}{\mathsf{\,x\,}}
\newcommand{\y}{\mathsf{\,y\,}}
\newcommand{\unit}{\mathsf{unit\,}}
\newcommand{\letc}{\mathsf{let \,}}
\newcommand{\inc}{\mathsf{\, in \,}}
\newcommand{\toc}{\mathsf{\, to \,}}
\newcommand{\rand}{\mathsf{rand\,}}
\newcommand{\coin}{\mathsf{coin\,}}
\newcommand{\inl}{\mathsf{inl\,}}
\newcommand{\inr}{\mathsf{inr\,}}
\newcommand{\case}[3]{\mathsf{case} \, #1 \, \mathsf{of} \, #2 \, | \, #3 \,}
\newcommand{\1}{\mathsf{\,\#1\,}}
\newcommand{\2}{\mathsf{\,\#2\,}}

% cbpv
\newcommand{\force}{\mathsf{force \,}}
\newcommand{\be}{\mathsf{\, be \,}}
\newcommand{\thunk}{\mathsf{thunk \,}}
\newcommand{\produce}{\mathsf{produce \,}}
\newcommand{\pmc}{\mathsf{\, pm \,}}
\newcommand{\as}{\mathsf{\, as \,}}
\newcommand{\B}{\underline{B}}

% translation
\newcommand{\semiid}[1]{[\![ #1 {]\!]}_{\mathsf{IID}}}
\newcommand{\sempc}[1]{[\![ #1 {]\!]}_{\mathsf{PC}}}

\geometry{letterpaper, total={7.5in, 10in}, left=.5in, top=.5in}

\begin{document}
%%%%%%%%%%%%%%%%%%%%%%%%%%%%%%%%%%%%%%%%%%%%%%%%%%
	\hrule
	\smallskip
	\hrule\hrule\hrule
	\medskip
	\noindent
	{\huge \bf Probabilistic Lambda Calculi}
	\medskip
	\newline
	{\huge \it Research Project}

	\hrule
	\medskip
	\noindent
	{\Large \bf Pedro Amorim and Eric Jackson}
	\medskip
	\hrule\hrule\hrule
	\smallskip
	\hrule
%%%%%%%%%%%%%%%%%%%%%%%%%%%%%%%%%%%%%%%%%%%%%%%%%%

\section{Abstract}

In section 2, we introduce a probabilistic version of Call-By-Push Value.
In section 3, we introduce two probabilistic lambda-calculi,
$\lambda_{IID}$ and $\lambda_{PC}$, which correspond to 
independent and identically distributed sampling 
and perfectly correlated sampling, respectively.

\section{Probabilistic Call-By-Push-Value}
\subsection{Syntax}
\subsubsection*{Types}
	CBPV types are given by $\tau = A \,|\, B$
	where $A$ and $B$ are defined inductively according to the following rules:
	\begin{equation*}
		\begin{split}
			A &::= 
			U \B
			\,|\, A + A
			\,|\, A \times A
			\,|\, 1			
			\,|\, \R \\
			\B &::=
			F A
			\,|\, \B \times \B
			\,|\, A \to \B \\
		\end{split}
	\end{equation*}
	As discussed in Levy's thesis,
	$A$ can be thought of as defining values
	and $B$ computations.
	
\subsubsection*{Expressions}
	\begin{equation*}
		\begin{split}
			V, M ::=\,
			 & \x \\
			|& () \\
			|& \lambda \x.\, M \\
			|& \letc \x \be V.\, M \\
			|& V` M \\
			|& \produce V \\
			|& M \toc \x.\, N \\
			|& \thunk M \\
			|& \force V \\
			|& (M, M) \\
			|& \1 M \\
			|& \2 M \\
			|& \pmc V \as (\!\x\!,\y\!).\, M \\
			|& \inl_{A + A} V \\
			|& \inr_{A + A} V \\
			|& \pmc V \as \{\!\x\!. M, \y\!. M\} \\
			|& \coin \\
			|& \rand \\
		\end{split}
	\end{equation*}

\newpage
\subsection{Static Semantics}
\begin{multicols}{2}
	\subsubsection*{Unit}
	\begin{equation*}
		\inferrule{ }{\Gamma \vdash^v () : 1}
	\end{equation*}

	\subsubsection*{Functions and Application}
	\begin{equation*}
		\inferrule{\Gamma_{\x : A} \vdash^v M : \B}{\Gamma \vdash^c \lambda \x.\, M : A \to \B} 
	\end{equation*}
	\begin{equation*}
		\inferrule{\Gamma \vdash^v V : A \\ \Gamma_{V : A} \vdash^c M : A \to \B}
		{\Gamma \vdash^c V` M : \B} 
	\end{equation*}
	\begin{equation*}
		\inferrule{\Gamma \vdash^v V : A \\ \Gamma_{V : A} \vdash^c M : A \to \B}
		{\Gamma \vdash^c \letc \x \be V.\, M : \B} 
	\end{equation*}
	
	\subsubsection*{Produce and To}
	\begin{equation*}
		\inferrule{\Gamma \vdash^v V : A}{\Gamma \vdash^c \produce V : F A} 
	\end{equation*}
	\begin{equation*}
		\inferrule{\Gamma \vdash^c M : F A \\ \Gamma_{\x : A} \vdash^c N : \B}
		{\Gamma \vdash^c M \toc \x.\, N : \B} 
	\end{equation*}
	
	\subsubsection*{Thunk and Force}
	\begin{equation*}
		\inferrule{\Gamma \vdash^c M : \B}{\Gamma \vdash^v \thunk M : U \B} 
	\end{equation*}
	\begin{equation*}
		\inferrule{\Gamma \vdash^v V : U \B}{\Gamma \vdash^c \force V : \B} 
	\end{equation*}

	\subsubsection*{Products}
	\begin{equation*}
		\inferrule{\Gamma \vdash^c M: \B \\ \Gamma \vdash^c M' : \B'}
		{\Gamma \vdash^c (M, M') : \B \times \B'} 
	\end{equation*}
	\begin{equation*}
		\inferrule{\Gamma \vdash^c M: \B \times \B'}{\Gamma \vdash^c \1 M : \B} 
	\end{equation*}
	\begin{equation*}
		\inferrule{\Gamma \vdash^c M: \B \times \B'}{\Gamma \vdash^c \1 M : \B'} 
	\end{equation*}
	\begin{equation*}
		\inferrule{\Gamma \vdash^c (M, M') : \B \times \B' \\ 
		\Gamma_{\x : \B, \y : \B'} \vdash^c M : \B}
		{\Gamma \vdash^c \pmc V \as (\!\x\!,\y\!).\, M : \B} 
	\end{equation*}
	
	\subsubsection*{Sums}
	\begin{equation*}
		\inferrule{\Gamma \vdash^v V: A}
		{\Gamma \vdash^v \inl_{A + A'} V : \tau_1 + \tau_2} 
	\end{equation*}
	\begin{equation*}
		\inferrule{\Gamma \vdash^v V: A}
		{\Gamma \vdash^v \inr_{A + A'} V : \tau_1 + \tau_2} 
	\end{equation*}
	\begin{equation*}
		\inferrule{\Gamma \vdash^v V : A + A' \\ 
		\Gamma_{\x : A, \y : A'} \vdash^c M : \B}
		{\Gamma \vdash^c \pmc V \as \{\!\x\!. M, \y\!. M\} : \B} 
	\end{equation*}

	\subsubsection*{Random Variables}
	\begin{equation*}
		\inferrule{ }{\Gamma \vdash^c \coin : F \R}
	\end{equation*}
	\begin{equation*}
		\inferrule{ }{\Gamma \vdash^c \rand : F \R}
	\end{equation*}
\end{multicols}

\subsection{Denotational Semantics}

\newpage
\section{$\lambda_{\iid}$ and $\lambda_{\pc}$}

\subsection{Syntax}

\subsubsection*{Types}
	The types of $\lambda_{\iid}$ and $\lambda_{\pc}$ are defined according to the following rules
	\begin{equation*}
		\tau ::=
		\unit
		\,|\, \R
		\,|\, \tau \to \tau
		\,|\, \tau + \tau
		\,|\, \tau \times \tau
	\end{equation*}
\subsubsection*{Expressions}
	The expressions of  $\lambda_{\iid}$ and $\lambda_{\pc}$ of the following form
	\begin{equation*}
		\begin{split}
			e ::=\,
			 & x \\
			|& () \\
			|& \lambda \x\! : \tau.\, e \\
			|& \letc \x = e \inc e \\
			|& e\, e \\
			|& \coin \\
			|& \rand \\
			|& \inl_{\tau_1 + \tau_2} e \\
			|& \inr_{\tau_1 + \tau_2} e \\
			|& \case{e}{e}{e} \\
			|& (e,\, e) \\
			|& \1 e \\
			|& \2 e \\
			|& e \toc \x \inc e \\
		\end{split}
	\end{equation*}

\subsection{Static Semantics}

\newpage
\section{Translating $\lambda_{\iid}$ and $\lambda_{\pc}$ to CBPV}
Rather than defining denotational semantics for $\lambda_{\iid}$ and $\lambda_{\pc}$,
we will define a translation from each of these languages into CBPV.
Then, we can use the CBPV semantics to generate denotational semantics.

\subsection{Types}
We begin with defining type translations from $\lambda_{\iid}$ and $\lambda_{\pc}$ to CBVP.
These translations are essentially identical to the translations of CBN and CBV to CBPV presented in Levy's thesis.


\begin{multicols}{2}
\subsubsection*{$\lambda_\iid$}
\hrule\hrule
\medskip
	\begin{equation*}
		\begin{split}
			\semiid{\unit} &\triangleq F 1 \\
			\semiid{\R} &\triangleq F \R \\
			\semiid{\tau_1 \to \tau_2} &\triangleq (U \semiid{\tau_1}) \to \semiid{\tau_2} \\
			\semiid{\tau_1 + \tau_2} &\triangleq F(U \semiid{\tau_1} + U \semiid{\tau_2}) \\
			\semiid{\tau_1 \times \tau_2} &\triangleq \semiid{\tau_1} \times \semiid{\tau_2} \\
		\end{split}
	\end{equation*}
		
\subsubsection*{$\lambda_\pc$}
\hrule\hrule
\medskip
	\begin{equation*}
		\begin{split}
			\sempc{\unit} &\triangleq 1 \\
			\sempc{\R} &\triangleq \R \\
			\sempc{\tau_1 \to \tau_2} &\triangleq U (\sempc{\tau_1} \to F\sempc{\tau_2}) \\
			\sempc{\tau_1 + \tau_2} &\triangleq U\sempc{\tau_1} + \sempc{\tau_2} \\
			\sempc{\tau_1 \times \tau_2} &\triangleq U(F\sempc{\tau_1} \times F\sempc{\tau_2}) \\
		\end{split}
	\end{equation*}
\end{multicols}

\subsection{Expressions}
Now, we define translations from expressions in $\iid$ and $\pc$
to CBPV.

% CBN
\subsubsection*{$\iid$}
\hrule\hrule
\medskip
	\begin{equation*}
		\begin{split}
			\semiid{x} &\triangleq \force x \\
			\semiid{\lambda x.\, e} &\triangleq \lambda x.\, \semiid{e} \\
			\semiid{\letc x = e_1 \inc  e_2} &\triangleq \letc x \be \thunk \semiid{e_1}.\, \semiid{e_2}  \\
			\semiid{e_1\, e_2}&\triangleq (\thunk \semiid{e_2})`\, \semiid{e_1}\\
			\semiid{\coin} &\triangleq \coin \\
			\semiid{\rand} &\triangleq \rand \\
			\semiid{\inl_{\tau_1 + \tau_2} e} &\triangleq \produce \inl \thunk \semiid{e} \\
			\semiid{\inr_{\tau_1 + \tau_2} e} &\triangleq \produce \inr \thunk \semiid{e} \\
			\semiid{\case{e_1}{e_2}{e_3}} &\triangleq	\semiid{e_1} \toc z.\, \pmc z \as 
				\{\inl x.\, \semiid{e_2}, \inr x.\, \semiid{e_3}\} \\
			\semiid{(e_1, e_2)} &\triangleq \\
			\semiid{\1 e} &\triangleq \\
			\semiid{\2 e} &\triangleq \\
			\semiid{e_1 \toc x \inc e2} &\triangleq \letc x = \sempc{e_1} \inc \semiid{e_2} \\
		\end{split}
	\end{equation*}
	
% CBV
\subsubsection*{$\pc$}	
\hrule\hrule
\medskip
	\begin{equation*}
		\begin{split}
			\sempc{x} &\triangleq  \force x \\
			\mathcal{T} [\![\lambda x.\, e{\rrb}_{PC} &\triangleq 
			\text{\bf produce thunk } \lambda x.\, \mathcal[\![ e {\rrb}_{PC} \\
			\mathcal{T} [\![\text{\bf let} x = e_1 \text{ \bf in } e_2{\rrb}_{PC} &\triangleq 
			\mathcal{T}[\![e_1{\rrb}_{PC} \text{ \bf to } x.\, \mathcal{T}[\![e_2{\rrb}_{PC}\\
			\mathcal{T} [\![e_1\, e_2{\rrb}_{PC} &\triangleq 
			\mathcal{T}[\![e_2{\rrb}_{PC} \text{ \bf to } x.\, \mathcal{T}[\![e_1{\rrb}_{PC} \text{ \bf to } f.\, x` (\text{\bf force } f)\\
			\mathcal{T} [\![\text{\bf coin}{\rrb}_{PC} &\triangleq 
			\text{\bf force coin} \\
			\mathcal{T} [\![\text{\bf rand}{\rrb}_{PC} &\triangleq
			\text{\bf force rand} \\
			\mathcal{T} [\![\text{\bf inl}_{\tau_1 + \tau_2} e{\rrb}_{PC} &\triangleq
			\mathcal{T}[\![e{\rrb}_{PC} \text{ \bf to } z.\, \text{\bf produce inl } z \\
			\mathcal{T} [\![\text{\bf inr}_{\tau_1 + \tau_2} e{\rrb}_{PC} &\triangleq
			\mathcal{T}[\![e{\rrb}_{PC} \text{ \bf to } z.\, \text{\bf produce inr } z \\
			\mathcal{T} [\![\text{\bf case } e_1 \text{\bf of } e_2 | e_3{\rrb}_{PC} &\triangleq
			\mathcal{T}[\![ e_1{\rrb}_{PC} \text{ \bf to } z. \text{ \bf pm } z \text{ \bf as } 
			\{ \text{\bf inl } x. \mathcal{T}[\![ e_2{\rrb}_{PC}, \text{ \bf inr } x. \mathcal{T}[\![ e_3{\rrb}_{PC}\} \\
			\mathcal{T} [\![(e_1, e_2){\rrb}_{PC} &\triangleq \\
			\mathcal{T} [\![\text{\bf \#1 } e{\rrb}_{PC} &\triangleq \\
			\mathcal{T} [\![\text{\bf \#2 } e{\rrb}_{PC} &\triangleq \\
			\mathcal{T} [\![e_1 \text{ \bf to } x \text{ \bf in } e2{\rrb}_{PC} &\triangleq \\
		\end{split}
	\end{equation*}

\begin{theorem} $\,$

	\begin{itemize}
		\item
		If $\Gamma \vdash_{\iid} e : \tau$ 
		then $\semiid{\Gamma} \vdash_c \semiid{e} : \semiid{\tau}$.
		\item
		If $\Gamma \vdash_{\iid} e : \tau$ 
		then $\sempc{\Gamma} \vdash_c \sempc{e} : \sempc{\tau}$.
	\end{itemize}
\end{theorem}

\begin{proof}$\,$

First, assume $\Gamma \vdash_\iid e : \tau$.
We proceed by mutual induction on $e$.
% TODO
\begin{itemize}
	\item {\bf Case:} $e = x$.
	First, $\semiid{e} = \force x$.
	For some $\tau$, we have $\Gamma \vdash_{\iid} e : \tau$.
	Consider the possible cases for $\tau$:
	\begin{itemize}
		\item $\tau = \unit$.
		\item $\tau = \R$.
		\item $\tau = \tau_1 \to \tau_2$.
		\item $\tau = \tau_1 + \tau_2$.
		\item $\tau = \tau_1 \times \tau_2$
	\end{itemize}
	
	\item {\bf Case:} $e = \lambda x : \tau .\, e'$.
	By inspection, $\Gamma \vdash_{\iid} (\lambda x : \tau .\, e') : \tau \to \tau'$ 
	for some $\tau'$ such that $\Gamma_{x : \tau} \vdash_{\iid}  e' : \tau'$.
	Then $\semiid{e} = \lambda x.\, \semiid{e'}$
	and $\semiid{\tau \to \tau'} = (U \semiid{\tau}) \to \semiid{\tau'}$.
	By induction, $\semiid{\Gamma_{x : \tau}} \vdash_c \semiid{e'} : \semiid{\tau'}$.
	
	\item {\bf Case:} $e = (\letc x = e_1 \inc e_2)$.
	
	\item {\bf Case:} $e = e_1\, e_2$.
	
	\item {\bf Case:} $e = \rand\!$.
	By definition, $\Gamma \vdash_\iid \rand : \R$,
	$\semiid{\rand} = \rand$ and $\semiid{\R} = F \R$.
	Then $\semiid{\Gamma} \vdash_c \semiid{\rand\!} : \semiid{\R}$.
	
	\item {\bf Case:} $e = \coin\!$.
	By definition, $\Gamma \vdash_\iid \coin : \R$,
	$\semiid{\coin} = \coin$ and $\semiid{\R} = F \R$.
	Then $\semiid{\Gamma} \vdash_c \semiid{\coin\!} : \semiid{\R}$.
	
	\item {\bf Case:} $e = \inl_{\tau_1 + \tau_2}\, e$.
	By inspection, $\Gamma \vdash_\iid (\inl_{\tau_1 + \tau_2}\, e) : \tau_1$.
	Since $\semiid{\inl_{\tau_1 + \tau_2}\, e} =  \produce \inl \thunk \semiid{e}$,
	
	\item {\bf Case:} $e = \inr_{\tau_1 + \tau_2}\, e$.
	
	\item {\bf Case:} $e = (e_1, e_2)$.
	
	\item {\bf Case:} $e = {\bf \#1}\, e$.
	
	\item {\bf Case:} $e = {\bf \#2}\, e$.
	
	\item {\bf Case:} $e = e_1 \toc x \inc e_2$.
	
\end{itemize}

Now, assume $\Gamma \vdash_{\pc} e : \tau$.
% TODO
\begin{itemize}
	\item {\bf Case:} $e = x$.

	\item {\bf Case:} $e = \lambda x : \tau .\, e'$.
	
	\item {\bf Case:} $e = (\letc x = e_1 \inc e_2)$.
	
	\item {\bf Case:} $e = e_1\, e_2$.
	
	\item {\bf Case:} $e = \rand\!$.
	
	\item {\bf Case:} $e = \coin\!$.
	
	\item {\bf Case:} $e = \inl_{\tau_1 + \tau_2}\, e$.
	
	\item {\bf Case:} $e = \inr_{\tau_1 + \tau_2}\, e$.
	
	\item {\bf Case:} $e = (e_1, e_2)$.
	
	\item {\bf Case:} $e = {\bf \#1}\, e$.
	
	\item {\bf Case:} $e = {\bf \#2}\, e$.
	
	\item {\bf Case:} $e = e_1 \toc x \inc e_2$.
	
\end{itemize}

\end{proof}

\begin{theorem}
	If $\Gamma \vdash_{\iid} e : \tau_1 \times \tau_2$
	then $\llb e \rrb = \mu_1 \times \mu_2$.
\end{theorem}

\begin{proof}
TODO!
\end{proof}

\section{Potential Applications}
	\subsection{System Security}
	% CBN : protocol generate secret.. 
	% IID. CBV : agent.. one key for whole computation.
	\subsection{Key Reuse}
	
	\subsection{Psuodo-Number Generators}
	
	\subsection{Random Variables}

\end{document}