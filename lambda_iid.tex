\documentclass{article}

\usepackage{geometry}
\usepackage{amsmath}
\usepackage{amssymb}
\usepackage{amsthm}
\usepackage{mathpartir}
\usepackage{multicol}

\newtheorem*{theorem}{Theorem}
\newtheorem*{prop}{Proposition}
\newtheorem*{definition}{Definition}
\newtheorem*{lemma}{Lemma}
\newtheorem*{example}{Example}
\newtheorem*{remark}{Remark}
\newtheorem*{claim}{Claim}

\renewcommand\qedsymbol{\bf QED}

\geometry{letterpaper, total={7.5in, 10in}, left=.5in, top=.5in}

\setcounter{secnumdepth}{0}

\begin{document}
%%%%%%%%%%%%%%%%%%%%%%%%%%%%%%%%%%%%%%%%%%%%%%%%%%
	\hrule
	\smallskip
	\hrule\hrule\hrule
	\medskip
	\noindent
	{\huge \bf Independently and Identically Distributed}
	\medskip
	\newline
	{\huge \it $\lambda_{IID}$}
	\smallskip
	\hrule
	\medskip
	\noindent
	{\Large \bf Eric Jackson}
	\medskip
	\hrule\hrule\hrule
	\smallskip
	\hrule
%%%%%%%%%%%%%%%%%%%%%%%%%%%%%%%%%%%%%%%%%%%%%%%%%%

% Syntax %%%%%%%%%%%%%%%%%%%%%%%%%%%%%%%%%%%%%%%%%
\section{Syntax}

TODO

% Static Semantics %%%%%%%%%%%%%%%%%%%%%%%%%%%%%%%
\section{Static Semantics}

TODO

% Dynamic Semantics %%%%%%%%%%%%%%%%%%%%%%%%%%%%%%
\section{Dynamic Semantic}

% Small Step %%%%%%%%%%%%%%%%%%%%%%%%%%%%%%%%%%%%%
	\subsection{Small-Step Semantics}
		\begin{multicols}{2}
% Context 
			\begin{equation*}
				\inferrule{
				\inferrule*[right={\sc Context}]
				{\langle e,\, n,\, m \rangle \rightarrow \langle e',\, n',\, m' \rangle }
				{\langle E(e),\, n,\, m\rangle \rightarrow \langle E(e'),\, n',\, m' \rangle}}
				{}
			\end{equation*}
		
% Beta-Reduction
			\begin{equation*}
				\inferrule{
				\inferrule*[right={\sc $\beta$-Reduction}]
				{\,}
				{\langle(\lambda x : \tau.\, e)\, e_2,\, n,\, m\rangle \rightarrow \langle e\{e_2/x\},\, n,\, m \rangle}}
				{}
			\end{equation*}

% Let
			\begin{equation*}
				\inferrule{
				\inferrule*[right={\sc Let}]
				{\,}
				{\langle\text{\bf let } x = e_1 \text{ \bf in } e_2,\, n,\, m\rangle \rightarrow \langle e_2\{e_1/x\},\, n,\, m \rangle}}
				{}
			\end{equation*}

% Binary Operations
			\begin{equation*}
				\inferrule{
				\inferrule*[right={\sc Bop}]
				{r_1 \bar{\oplus} r_2 = r }
				{\langle r_1 \oplus r_2,\, n,\, m\rangle \rightarrow \langle r,\, n,\, m \rangle}}
				{}
			\end{equation*}
		
% Coin
			\begin{equation*}
				\inferrule{
				\inferrule*[right={\sc Coin}]
				{\,}
				{\langle {\bf coin},\, n,\, m\rangle \rightarrow \langle \text{\bf hd } n,\, \text{\bf tl } n,\, m \rangle}}
				{}
			\end{equation*}

% Rand
				\begin{equation*}
				\inferrule{
				\inferrule*[right={\sc Rand}]
				{\,}
				{\langle {\bf rand},\, n,\, m\rangle \rightarrow \langle \text{\bf hd } m,\, n,\, \text{\bf tl } m \rangle}}
				{}
			\end{equation*}

% Case-Left
				\begin{equation*}
				\inferrule{
				\inferrule*[right={\sc Case-Left}]
				{\,}
				{\langle (\text{\bf case } {\bf inl}_{\tau_1 + \tau_2} e \text{ \bf of } e_2 \,|\, e_3) ,\, n,\, m\rangle \rightarrow \langle e_2\, e,\, n,\, m \rangle}}
				{}
			\end{equation*}
			
% Case-Right
				\begin{equation*}
				\inferrule{
				\inferrule*[right={\sc Case-Right}]
				{\,}
				{\langle (\text{\bf case } {\bf inr}_{\tau_1 + \tau_2} e \text{ \bf of } e_2 \,|\, e_3) ,\, n,\, m\rangle \rightarrow \langle e_3\, e,\, n,\, m \rangle}}
				{}
			\end{equation*}
			
% Project-1
				\begin{equation*}
				\inferrule{
				\inferrule*[right={\sc Proj-1}]
				{\,}
				{\langle \text{\bf \#1 } (e_1, e_2) ,\, n,\, m\rangle \rightarrow \langle e_1,\, n,\, m \rangle}}
				{}
			\end{equation*}
			
% Project-2
				\begin{equation*}
				\inferrule{
				\inferrule*[right={\sc Proj-2}]
				{\,}
				{\langle \text{\bf \#2 } (e_1, e_2) ,\, n,\, m\rangle \rightarrow \langle e_2,\, n,\, m \rangle}}
				{}
			\end{equation*}
			
% To
			\begin{equation*}
				\inferrule{
				\inferrule*[right={\sc To-pc}]
				{\,}
				{\langle v_1 \text{ \bf to } x \text{ \bf in } e_2, n, m \rangle \to 
				\langle e_2 \{v_1 / x\}, n, m \rangle}}
				{}
			\end{equation*}
		\end{multicols}
			
% Big Step %%%%%%%%%%%%%%%%%%%%%%%%%%%%%%%%%%%%%%%
	\subsection{Big-Step Semantics}
% Beta-Reduction
			\begin{equation*}
				\inferrule{
				\inferrule*[right={\sc $\beta$-Reduction}]
				{\langle e_1, n, m \rangle \Downarrow \langle \lambda x.\, e_1', n', m'\rangle \\
				\langle e_1'\{e_2/x\}, n',m'\rangle \Downarrow \langle v, n'', m'' \rangle}
				{\langle e_1\, e_2,\, n,\, m\rangle \Downarrow \langle v,\, n'',\, m'' \rangle}}
				{}
			\end{equation*}

% Let
			\begin{equation*}
				\inferrule{
				\inferrule*[right={\sc Let}]
				{\langle e_2\{e_1/x\}, n, m \rangle \Downarrow \langle v, n', m'\rangle}
				{\langle\text{\bf let } x = e_1 \text{ \bf in } e_2,\, n,\, m\rangle \Downarrow \langle v,\, n',\, m' \rangle}}
				{}
			\end{equation*}

% Binary Operations
			\begin{equation*}
				\inferrule{
				\inferrule*[right={\sc Bop}]
				{\langle e_1, n, m \rangle \Downarrow \langle r_1, n', m' \rangle \\
				\langle e_2, n', m' \rangle \Downarrow \langle r_2, n'', m'' \rangle \\
				 r_1 \bar{\oplus} r_2 = r }
				{\langle e_1 \oplus e_2,\, n,\, m\rangle \Downarrow \langle r,\, n'',\, m'' \rangle}}
				{}
			\end{equation*}
% Coin
			\begin{equation*}
				\inferrule{
				\inferrule*[right={\sc Coin}]
				{\,}
				{\langle {\bf coin},\, n,\, m\rangle \Downarrow \langle \text{\bf hd } n,\, \text{\bf tl } n,\, m \rangle}}
				{}
			\end{equation*}

% Rand
				\begin{equation*}
				\inferrule{
				\inferrule*[right={\sc Rand}]
				{\,}
				{\langle {\bf rand},\, n,\, m\rangle \Downarrow \langle \text{\bf hd } m,\, n,\, \text{\bf tl } m \rangle}}
				{}
			\end{equation*}
			
% Case-Left
				\begin{equation*}
				\inferrule{
				\inferrule*[right={\sc Case-Left}]
				{\langle e_2\, e_1, n, m \rangle \Downarrow \langle v, n', m' \rangle}
				{\langle (\text{\bf case } {\bf inl}_{\tau_1 + \tau_2} e_1 \text{ \bf of } e_2 \,|\, e_3) ,\, n,\, m\rangle \Downarrow \langle v,\, n',\, m' \rangle}}
				{}
			\end{equation*}

% Case-Right
				\begin{equation*}
				\inferrule{
				\inferrule*[right={\sc Case-Left}]
				{\langle e_3\, e_1, n, m \rangle \Downarrow \langle v, n', m' \rangle}
				{\langle (\text{\bf case } {\bf inr}_{\tau_1 + \tau_2} e_1 \text{ \bf of } e_2 \,|\, e_3) ,\, n,\, m\rangle \Downarrow \langle v,\, n',\, m' \rangle}}
				{}
			\end{equation*}

% Project-1
				\begin{equation*}
				\inferrule{
				\inferrule*[right={\sc Proj-1}]
				{\langle e_1, n, m \rangle \Downarrow \langle v, n', m' \rangle}
				{\langle \text{\bf \#1 } (e_1, e_2) ,\, n,\, m\rangle \Downarrow \langle v, n', m' \rangle}}
				{}
			\end{equation*}
			
% Project-2
				\begin{equation*}
				\inferrule{
				\inferrule*[right={\sc Proj-2}]
				{\langle e_2, n, m \rangle \Downarrow \langle v, n', m' \rangle}
				{\langle \text{\bf \#2 } (e_1, e_2) ,\, n,\, m\rangle \Downarrow \langle v, n', m' \rangle}}
				{}
			\end{equation*}
			
% To
			\begin{equation*}
				\inferrule{
				\inferrule*[right={\sc To-pc}]
				{\langle e_2\{(\lambda y.\, e_1) () / x\}, n, m \rangle \Downarrow \langle v, n', m' \rangle}
				{\langle e_1 \text{ \bf to } x \text{ \bf in } e_2, n, m \rangle  
				\Downarrow \langle v, n', m' \rangle}}
				{}
			\end{equation*}


% Denotational %%%%%%%%%%%%%%%%%%%%%%%%%%%%%%%%%%%
	\subsection{Denotational Semantics}
		\begin{equation*}
		\begin{split}
			[\![r]\!] &\triangleq \delta_r \\
			[\![\lambda x : \tau .\, e]\!] &\triangleq x \mapsto [\![e]\!] \\
			[\![\text{\bf let } x = e_1 \text{ \bf in } e_2]\!] &\triangleq [\![e_2\{e_1 / x\}]\!] \\
			[\![e_1 \text{ } e_2]\!] &\triangleq [\![e_1]\!] \circ [\![e_2]\!] \\
			[\![{\bf coin}]\!] &\triangleq \frac{1}{2}(\delta_0 + \delta_1) \\
			[\![{\bf rand}]\!] &\triangleq \lambda \\
		\end{split}
	\end{equation*}
	
% Translation %%%%%%%%%%%%%%%%%%%%%%%%%%%%%%%%%%%%
\subsection{Translation to CBPV}
	\begin{equation*}
		\begin{split}
			\mathcal{T} [\![x]\!] &\triangleq 
			\text{\bf force } x \\
			\mathcal{T} [\![\lambda x.\, e]\!] &\triangleq 
			\lambda x.\, \mathcal[\![ e ]\!] \\
			\mathcal{T} [\![\text{\bf let} x = e_1 \text{ \bf in } e_2]\!] &\triangleq 
			\text{\bf let } x \text{ \bf be } \mathcal{T}[\![e_1]\!].\, \mathcal{T}[\![e_2]\!]\\
			\mathcal{T} [\![e_1\, e_2]\!] &\triangleq 
			(\text{\bf thunk} \mathcal{T}[\![e_2]\!])`\, \mathcal{T}[\![e_1]\!]\\
			\mathcal{T} [\![\text{\bf coin}]\!] &\triangleq
			\text{\bf coin} \\
			\mathcal{T} [\![\text{\bf rand}]\!] &\triangleq
			\text{\bf rand} \\
			\mathcal{T} [\![\text{\bf inl}_{\tau_1 + \tau_2} e]\!] &\triangleq
			\text{\bf produce inl thunk } \mathcal{T}[\![e]\!] \\
			\mathcal{T} [\![\text{\bf inr}_{\tau_1 + \tau_2} e]\!] &\triangleq
			\text{\bf produce inr thunk } \mathcal{T}[\![e]\!] \\
			\mathcal{T} [\![\text{\bf case } e_1 \text{\bf of } e_2 | e_3]\!] &\triangleq	
			\mathcal{T}[\![ e_1]\!] \text{ \bf to } z. \text{ \bf pm } z \text{ \bf as } 
			\{\text{\bf inl } x. \mathcal{T}[\![ e_2]\!], 
			\text{ \bf inr } x. \mathcal{T}[\![ e_3]\!]\} \\
			\mathcal{T} [\![(e_1, e_2)]\!] &\triangleq \\
			\mathcal{T} [\![\text{\bf \#1 } e]\!] &\triangleq \\
			\mathcal{T} [\![\text{\bf \#2 } e]\!] &\triangleq \\
			\mathcal{T} [\![e_1 \text{ \bf to } x \text{ \bf in } e2]\!] &\triangleq \\
		\end{split}
	\end{equation*}

\end{document}